\documentclass[12pt]{article}
\RequirePackage[OT1]{fontenc}
\RequirePackage[leqno,cmex10]{amsmath}
\RequirePackage{amsthm}
\RequirePackage[numbers]{natbib}
\RequirePackage[colorlinks,citecolor=blue,urlcolor=blue]{hyperref}

\usepackage{mathrsfs,dsfont}
\usepackage{latexsym}
\usepackage{amssymb}
%\usepackage{epsfig}
%\usepackage{fullpage}
%\usepackage{setspace}
\usepackage[english]{babel}
\usepackage[latin1]{inputenc}
\usepackage{color}
\usepackage{amsfonts}
\usepackage{bbm}
\usepackage{enumerate}
\RequirePackage{xspace}
\usepackage[ruled, section]{algorithm}
\usepackage{algpseudocode}
%\usepackage{algorithmicx}
%\usepackage{algorithm2e}
%\usepackage[paper=a4paper,dvips,top=1.5cm,left=1.5cm,right=1.5cm,
%    foot=1cm,bottom=1.5cm]{geometry}
\usepackage{subfigure}
\usepackage {graphicx}
\usepackage {pgf}
\usepackage{yhmath}
\usepackage[normalem]{ulem}

\usepackage{multido}
\usepackage{pstricks,pst-plot,pstricks-add,pst-math}

\pagestyle{empty}
\textwidth=6.5in
\parindent=0in
\textheight=9.5in
\hoffset=-.65in
\voffset=-1.05in
\def\Ans{\textcolor{blue}}

\usepackage[T3,T1]{fontenc}
\DeclareSymbolFont{tipa}{T3}{cmr}{m}{n}
\DeclareMathAccent{\inv}{\mathalpha}{tipa}{16}

%% Greek symbols
\let\ga=\alpha \let\gb=\beta \let\gc=\gamma \let\gd=\delta \let\gee=\epsilon
\let\gf=\varphi \let\gh=\eta \let\gi=\iota  \let\gk=\kappa \let\gl=\lambda \let\gm=\mu
\let\gn=\nu \let\go=\omega \let\gp=\pi \let\gr=\rho \let\gs=\sigma \let\gt=\tau \let\gth=\vartheta
\let\gx=\chi \let\gy=\upsilon \let\gz=\zeta
\let\gC=\Gamma \let\gD=\Delta \let\gF=\Phi \let\gL=\Lambda \let\gTh=\Theta
\let\gO=\Omega   \let\gP=\Pi    \let\gPs=\Psi  \let\gS=\Sigma \let\gU=\Upsilon \let\gX=\Chi
\let\gY=\Upsilon

\newcommand{\vx}{\mathbf{x}}


\begin{document}

\begin{center}
{\Large Notes on Models}\\
\vspace*{1cm}
\end{center}

We consider the model for disease spread and epidemic wave front starting from a single source point.

Let us consider that $\vx_0 \in \mathbf{R}^2$ is the source point. Let us consider that the disease incidence at point $\vx \in \mathbf{R}^2$ is given by
\begin{equation}
\label{eqn:model}
y(r, \phi, t|\gTh) = f(r, \phi, t| \gTh)
\end{equation}  
where, $\gTh$ are the parameters of the model and $(r, \phi)$ are the polar co-ordinates of the point $\vx$ with $\vx_0$ as the center.

The following PDE models are considered for $y$:
\begin{eqnarray}
\label{eqn:pde1}
\frac{\partial y}{\partial r} & = & -\frac{by(1-y)}{g(\phi) + r} \\ 
\label{eqn:pde2}
\frac{\partial y}{\partial t} & = & ay(1-y)
\end{eqnarray}
where, $g(\phi):[0, 2\pi] \rightarrow \mathbf{R}$ is a function of the angle between source and target; $a$ and $b$ are parameters of the model.

Solving \eqref{eqn:pde1}, we get,
\begin{equation}
\label{eqn:rmodel}
\log\left(\frac{y}{1-y}\right) = -b\log\left(1 + \frac{r}{g(\phi)}\right) + c(t, \phi)
\end{equation}
where, $c(t, \phi)$ is a normalizing function. Solving \eqref{eqn:pde2}, we get,
\begin{equation}
\label{eqn:tmodel}
\log\left(\frac{y}{1-y}\right) = at + d(r, \phi)
\end{equation}
where, $d(r, \phi)$ is a normalizing function. Now, equating \eqref{eqn:rmodel} and \eqref{eqn:tmodel}, we get that,
\begin{equation}
\label{eqn:rtmodel}
\log\left(\frac{y}{1-y}\right) = -b\log\left(1 + \frac{r}{g(\phi)}\right) + at + c(\phi)
\end{equation}
Now, if we consider a model for $\frac{\partial y}{\partial \phi}$, we can get a functional form for $c(\phi)$. But, for now, we are ignoring it.

Now, by simplifying \eqref{eqn:rtmodel}, we get that,
\begin{equation}
y(r, \phi, t | \gTh) = \frac{1}{1+ \left(1 + \frac{r}{g(\phi)}\right)^b\exp^{-at}A(\phi)}
\end{equation}
where, the parameter set $\gTh = \{a, b, g(\cdot)\}$ and $A(\phi) = \frac{1}{c(\phi)}$. This is a disease incidence spatio-temporal process with spatial kernel: $A(\phi)\left(1 + \frac{r}{g(\phi)}\right)^{-b}$ (same as the geometric kernel in Rieux et. al. (2014)).

Now, velocity,
\begin{eqnarray}
\nonumber
v & = & -\frac{\partial r}{\partial t} = \frac{a}{b}\left(\frac{1}{g(\phi) + r}\right)^{-1} \\
\label{eqn:vmodel}
 & = & M \left(\frac{1}{g(\phi) + r}\right)^{-1}
\end{eqnarray}
where, $M=\frac{a}{b}$ is a parameter. Integrating \eqref{eqn:vmodel}, we get that,
\begin{equation}
\label{eqn:fitmodel1}
\log\left(1 + \frac{r}{g(\phi)}\right) = -Mt + h(\phi)
\end{equation}
or,
\begin{equation}
\label{eqn:fitmodel2}
r = g(\phi)\left(1 - u(\phi)\exp^{-Mt}\right)
\end{equation}
where, $h(\phi)$ and $u(\phi)$ are normalizing function parameters.

Remember that we fit the models \eqref{eqn:fitmodel1} and \eqref{eqn:fitmodel2} using the data $\{(r_1, \phi_1, t_1), \ldots, (r_n, \phi_n, t_n)\}$.

\end{document}
